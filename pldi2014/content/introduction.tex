\section{Introduction}

% what is the problem?
Higher-order behavioral contracts can take inordinate time and space to enforce.

%\newcommand{\mllam}[3]{\lambda #1:\mathtt{#2}.\,#3}
%\newcommand{\mlif}[3]{\mathtt{if}\,#1\,\mathtt{then}\,#2\,\mathtt{else}\,#3}

%Consider, for instance, the program
%\[
%\mathtt{even?}=\mllam{n}{Int}{\mlif{n=0}{\mathtt{true}}{5}}\\
%\lambda n:\mathtt{Int}.if n=0 then true else odd? (n - 1)\\
%even?=\lambda n:Int.if n=0 then true else odd? (n - 1)
%\]
%But we expect many contract projections to be benign.

Consider, for instance, the program
\begin{schemedisplay}
(define/contract (fact n a)
  (nonnegative-integer/c positive-integer/c . -> . positive-integer/c)
  (if (= n 0)
      a
      (fact (- n 1) (* n a))))
\end{schemedisplay}
which attaches dynamic domain and range contracts to the tail-recursive factorial.
This contracted procedure behaves as if it were written
\begin{schemedisplay}
(define (fact n a)
  (positive-integer/c
   ((LAMBDA (n a)
      (if (= n 0)
          a
          (fact (- n 1) (* n a))))
    (nonnegative-integer/c n)
    (positive-integer/c a))))
\end{schemedisplay}
where \scheme{positive-integer/c} and \scheme{nonnegative-integer/c} are procedures which raise an error if their argument does not belong to the domain reflected by the procedure's name.
Written in this way, it is apparent that \scheme{fac} is not tail-recursive.
Instead, the context grows with every recursive call so that \scheme{(fact 6 1)} reduces as
\begin{schemedisplay}
(fact 6 1)
(positive-integer/c (fact 5 6))
(positive-integer/c (positive-integer/c (fact 4 30)))
...
(positive-integer/c ... (positive-integer/c (fact 0 720)))
(positive-integer/c ... (positive-integer/c 720))
\end{schemedisplay}
whence the result will be validated six times---and \scheme{n} times for \scheme{(fact n 1)}.
The attachment of these contracts has transformed a constant-space algorithm into a linear-space one and, while the time complexity hasn't suffered asymptotically, the accumulated checks represent a real (and, in this case, unnecessary) cost.
Thus, a simple contract ascription can upset our performance expectations for even simple procedures.

As contract specifications become more elaborate, they become more susceptible to error.
In contract systems, such as Racket's~\cite{plt-tr1}, which allow arbitrary predicate procedures written in the source language itself to serve as specifications, this error is effected by the programmer.
It is only natural, then, to contract these procedures as well, an act which aides error diagnosis but compounds the run-time cost of contracts.
In these systems, execution time of the program proper can be dwarfed by that of the enforcement of its specification (in one case accounting for 96\% of total execution time) \cite{strickland2012chaperones}.

The moderation of these costs has been studied extensively.
Herman et al.~\cite{herman2010space} observed the potential accumulation of casts in the blame calculus---similar in principle to the accumulation of guards by our contracted \scheme{fact} procedure---and applied the coercion calculus of Henglein~\cite{henglein1994dynamic} to compact them.
Siek and Wadler~\cite{siek2010threesomes} consider the same problem and apply their own solution of \emph{threesomes}.
Siek et al.~\cite{siek2006gradual} explicate type systems which allow \emph{gradual typing} in which they provide a low-cost path from dynamically- to statically-checked invariants.
Flanagan and Knowles~\cite{flanagan2006hybrid} introduce \emph{hybrid type checking} in which static analysis is used to establish contract compliance statically where possible and insert dynamic checks where not.
%Using this final approach, a disciplined programmer enjoys the early safety guarantees and performance characteristics of statically-verified code or suffers the cost of belated error detection of dynamically-checked specifications.

Unfortunately, modular development can impede these analyses because the greatest amount of knowledge a module can have about its imports \emph{is} their behavioral contract.\footnote{Indeed, it can be less than this, as when the behavioral specification is not codified entirely within the program.}
For example, recognizing that contracts constitute total formal knowledge, Tobin-Hochstadt and Van Horn \cite{tobin2012higher} conduct an intra-module analysis which identifies extra-module references with opaque values that satisfy their contract.
% also, contracts are attached to values that flow across module boundaries.
Nevertheless, static analyses typically perform best when given whole programs.
For instance, the whole-program ML compiler MLton \cite{weeks2006whole} is able to perform thorough inter-procedural and inter-modular analysis and effect significant program optimizations.
We similarly expect the establishment of certain properties to become more accessible as code knowledge increases.
%We expect the emergence of the possibility of more properties being proved about the code as more knowledge about it becomes available.

We turn our attention to whole-program analysis of Racket \cite{plt-tr1}, a higher-order, dynamically-typed language, for the purpose of reducing the run-time burden of contracts.
Racket programs are organized into modules which consist of import and export declarations, definitions, and expressions.
Because Racket has rich facilities for syntactic extension, execution of code must be carefully controlled, especially since modules can be compiled separately \cite{flatt2002composable}.
Therefore, the quickest way to a whole program is by \emph{demodularizing} pre-compiled modules into a whole program in the Racket bytecode language (which, contrary to its name, is a fairly rich higher-order language).
At this level, contracts are manifest in terms of \emph{interposition} primitives.

Recent advances in higher-order program analysis \cite{earl2012introspective} provide perfectly precise return flow and allow the form of the stack to be approximated at each control point in the program.
The original purpose of this approximation was to provide an abstract garbage collector a conservative set of live frames.
We repurpose this capability to analyze a notion of dependence similar to that of Might and Prabhu~\cite{might2009interprocedural} allowing us to judge the safe removal of dynamic contracts \emph{even when the installation of other contracts depends on them}.
The expressiveness of Racket's contract language admits dynamic checks which diverge, so we must integrate a termination analysis based on Sereni's~\cite{sereni2007termination} that informs the discharge of dependently-installed contracts.

We proceed as follows:
We discuss interposition and its relationship to behavioral contracts in more depth in Section 2.
We introduce the syntax and semantics of \chapcalc, a core calculus with facilities for interposing higher-order values, in Section 3.
We present two theorems which establish the conditions for the safe removal or weakening of interposition in Section 4.
We present a static analysis which can judge when these conditions hold in Section 5.
We discuss the influence of parameters of the analysis on precision and performance in Section 6.
We discuss some Racket language features which require more care to integrate with the present analysis in Section 7.
We conclude by discussing the future application and direction of this work in Section 8.