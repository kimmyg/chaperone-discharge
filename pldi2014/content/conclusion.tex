\section{Conclusion}

Interposition primitives exist for more than functions in Racket: boxes, vectors, hash tables, continuation mark keys, etc.

We omitted from \chapcalc a few primitives that exist in Racket that must be accounted for by our analysis.
A call \scheme{(chaperone-of? f g)} determines whether \scheme{f} is \scheme{equal?} to \scheme{g} modulo chaperones.
In other words, it determines whether \scheme{f} could be constructed from the parts of \scheme{g} and some chaperone installation.
This primitive presents no problem for chaperone discharge as \scheme{(chaperone-of? f f)} always evaluates to \scheme{#t}.
However, if a chaperone has been promoted to an impersonator and flows to the first argument position, the meaning of the program is changed, making the promotion unsound.
If chaperone values that both otherwise can and cannot be promoted flow to this position, we cannot promote the chaperones.
If only promoted or discharged chaperones flow to it, we can replace it with the corresponding \scheme{impersonator-of?} call.

We also must contend with the \scheme{chaperone?} and \scheme{impersonator?} predicates. We can apply similar reasoning to integrate them into our original analysis.
Given that Racket programmers are discouraged from using them, we don't expect them to regularly upend our analysis, but of course we must account for them.

%\section{Next}

%Not changing complexity, the time factor is less pernicious, but it is hard not to think of the waste effected by the application of a sequence of idempotent operators.

%Now the programmer wants to define his own contract predicates \scheme{even?} and \scheme{odd?}. Aware of the cost of contracts on tail-recursive functions, he opts against their use.
%\begin{schemedisplay}
%(define (even? n)
%  (if (= n 0)
%      #t
%      (odd? (- n 1))))
%      
%(define (odd? n)
%  (if (= n 1)
%      #t
%      (even? (- n 1))))
%\end{schemedisplay}
%for use in the definition
%\begin{schemedisplay}
%(define/contract (partial-succ n)
%  (odd? . -> . even?)
%  (+ n 1))
%\end{schemedisplay}
%which calculates the successor only for odd numbers.

%What is the result of the call \scheme{(partial-succ 0)}?
%The programmer hopes a contract violation will be detected with blame assigned to the caller.
%Instead, his program never halts.

%Now suppose the programmer recruits \scheme{odd?} and \scheme{even?} as contract predicates but, as happens with , fails to implement odd? correctly inst


%It takes time to determine compliance of first-order values as first-order contracts can be arbitrary predicates.
%Additionally, functions may ascribe contracts to their higher-order arguments which themselves do similarly, even recursively.

%There isn't much a calculus can do to solve the time issue, especially in the face of dependent contracts.

%Because contract compliance of higher-order values cannot be determined in general at ascription time, compliance checks must be attached to values to be performed later.
%These attachments may accumulate, even changing the asymptotic space complexity of the program \cite{herman2010space} unless otherwise handled (again see \cite{herman2010space} or, for example, \cite{siek2010threesomes}).

%This space issue is well-known and a few different approaches have been taken to deal with it:
%Herman's coercion calculus, Wadler and Sieks' threesome calculus, the space-efficient contracts author
%These do well to soundly preserve the asymptotic space complexity of the program and perhaps the time as well.




%The motivation for this was to provide an abstract garbage collector a set of live frames, disregarding any stack structure latent in the control flow graph.

%This description alone is enough to allow us to reason statically about features based on stack inspection, such as security checks [cite] and dynamic binding [cite].
%We will use it to approximate blame on a misbehaving program.
%After considering all possible sources of such blame, a straightforward analysis allows us to alter contract sites, shifting some of the dynamic burden to the static.

% thus, in order to soundly discharge contracts from the original program, we must soundly discharge impersonators and chaperones.
%The implementation of familiar language features is sometimes accomplished through alien constructs.
% (Because the feature must be situated particularly within the already existing features of the language.)
%Contracts are a common feature of dynamically-typed languages.
%Higher-order contracts \cite{findler2002contracts} increase the domain of contracts to higher-order values.
%Of course, codomain compliance of a higher-order contract cannot be decided at attachment time.
%Instead, contracts are decomposed and positioned to intercept the arguments and results of application--the earliest possible time that contract compliance can be determined.
%This \emph{de facto} wrapping of deferred contract checks imposes space and time overhead.
%Wadler and his cronies and Herman and his cronies \cite{wadler2009well,herman2010space} show how the accumulation of wrappers can grow unbounded and offer various solutions to ameliorate this.
%Determining contract compliance can make up a significant portion of reduction.

%Higher-order contracts \cite{findler2002contracts} are now familiar constructs of languages but their implementation in Racket, which must cope with other language features--perhaps incidentally, is in terms of value wrappers.
% \emph{impersonators} and \emph{chaperones}.
%In Racket, data contracts are implemented at the bytecode level using interposition primitives--that is, values are wrapped and these wrappers intercept arguments and results.
%Because functions are data in a higher-order language, special care has been given to enforcing safe and sound function contracts.
%Of course, codomain contract compliance cannot be decided.
%In consequence, codomain contract checks are deferred until the function actually delivers a value.

%Nevertheless, analysis tools have access to more of certain kinds of information after compilation and so have the potential to prove more and different properties about programs.
%Specifically, analysis tools for Racket have access to the whole program after module compilation through a \emph{demodularization} process.

%Racket program lifecycle--the tipping of various types of information

% a static analysis defines a class of program transformations
% if the analysis is sound, the transformation is meaning-preserving
% are these two preceding statements true?

%This goal can be sought at each of the various stages of the Racket program lifecycle.
%Initially, programs are organized into modules, and dynamic behavior can be gleaned from contract declarations at module boundaries.\cite{tobin2012higher}
%Each module is compiled in isolation to bytecode; the bytecode compiler performs optimizations which rely on at most basic flow analysis are performed (constant-folding, 
%``The bytecode compiler applies all standard optimizations, such as constant propagation, constant folding, inlining, and dead-code elimination.''
%The journey to this stage discards high-level information and doesn't acquire much new information, so analysis here is probably fruitless.

%Racket modules are compiled to bytecode. This

%implementation of contracts within racket using interposition

%talk about blame and the method Racket uses to assign blame--the compilation process, etc.